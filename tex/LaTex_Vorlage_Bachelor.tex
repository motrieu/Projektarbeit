\documentclass[12pt]{article}
\usepackage[utf8]{inputenc}
\usepackage{a4}
\usepackage{exscale,latexsym}
\usepackage{graphicx}
\usepackage{pdfpages}
\usepackage{epic}
\usepackage{setspace}
\onehalfspacing
\usepackage[flushmargin]{footmisc}
\usepackage{amsmath}
%\usepackage{europs}
\usepackage{german}
\usepackage{natbib}
\usepackage[small]{caption}
\usepackage{graphpap}
\usepackage{rotating}
\usepackage{amssymb}
%\usepackage{wasysym}
\usepackage[dvips]{epsfig}
\usepackage{multirow}
\usepackage{acronym}
\usepackage[a4paper,left=4cm, right=1.5cm,top=2.5cm, bottom=2.0cm]{geometry} %Seitenr�nder
\usepackage{fancyhdr}
\usepackage{setspace}
%%%%%%%%%%%%%%%%%%%%%%%%%%%%%%%%%%%
%
%  File        : math.tex
%  Author : Tobias Holicki
%  Date      : 24.11.2015
%
%%%%%%%%%%%%%%%%%%%%%%%%%%%%%%%%%%%
%
% This file contains basic math commands for tex and is based on files from
% Carsten W. Scherer and Matthias Fetzer.
%



%-----------------------------------------------------------------------
% Needed packages
%-----------------------------------------------------------------------
\usepackage{mathrsfs}  % for \mathscr
\DeclareMathAlphabet{\mathpzc}{T1}{pzc}{m}{it} % for \mathpzc
\usepackage{bm} % bold math symbols \bm{}
\usepackage{amsmath} % many math things...
\usepackage{amssymb} % symbols as \prec \succ etc
\usepackage{arydshln}% dashed horizontal and vertical lines
\usepackage{mathdots} % \iddots


%-----------------------------------------------------------------------
% Blackboard bold letters
%-----------------------------------------------------------------------
\newcommand{\set}[1]{\mathbb{#1}}
\newcommand{\C}{{\mathbb{C}}}         % complex numbers
\newcommand{\Q}{{\mathbb{Q}}}         % rational numbers
\newcommand{\R}{{\mathbb{R}}}         % real numbers
\newcommand{\Z}{{\mathbb{Z}}}         % integers
\newcommand{\N}{{\mathbb{N}}}         % natural numbers
\newcommand{\K}{{\mathbb{K}}}         % symbol for a field
\newcommand{\M}{{\mathbb{M}}}
\renewcommand{\H}{\mathbb{H}}		 % hermitian matrices
\renewcommand{\L}{\mathbb{L}}
\renewcommand{\S}{\mathbb{S}}          % symmetric matrices
\newcommand{\D}{\mathbb{D}}             % disc
\newcommand{\A}{\mathbb{A}}
\renewcommand{\P}{ {\mathbb{P}} }
\newcommand{\Rset}{\mathbb{R}}
\newcommand{\Cset}{\mathbb{C}}
\newcommand{\Sset}{\mathbb{S}}
\newcommand{\Kset}{\mathbb{K}}


%-----------------------------------------------------------------------
% Caligraphic letters
%-----------------------------------------------------------------------
\renewcommand{\c}[1]{\mathcal{#1}}
\renewcommand{\Ac}{\c{A}}
\newcommand{\Bc}{\c{B}}
\newcommand{\Cc}{\c{C}}
\newcommand{\Dc}{\c{D}}
\newcommand{\Ec}{\c{E}}
\newcommand{\Fc}{\c{F}}
\newcommand{\Gc}{\c{G}}
\newcommand{\Hc}{\c{H}}
\newcommand{\Kc}{\c{K}}
\newcommand{\Lc}{\c{L}}
\newcommand{\Pc}{\c{P}}
\newcommand{\Sc}{\c{S}}
\newcommand{\Tc}{\c{T}}
\newcommand{\Uc}{\c{U}}
\newcommand{\Vc}{\c{V}}
\newcommand{\Wc}{\c{W}}
\newcommand{\Xc}{\c{X}}
\newcommand{\Yc}{\c{Y}}
\newcommand{\Zc}{\c{Z}}


%-----------------------------------------------------------------------
% Script letters
%-----------------------------------------------------------------------
\newcommand{\As}{\mathscr{A}}
\newcommand{\Bs}{\mathscr{B}}
\newcommand{\Cs}{\mathscr{C}}
\newcommand{\Ds}{\mathscr{D}}
\newcommand{\Es}{\mathscr{E}}
\newcommand{\Fs}{\mathscr{F}}
\newcommand{\Gs}{\mathscr{G}}
\newcommand{\Ls}{\mathscr{L}}
\newcommand{\Ps}{\mathscr{P}}
\newcommand{\Xs}{\mathscr{X}}
\newcommand{\Ys}{\mathscr{Y}}


%-----------------------------------------------------------------------
% Bold letters
%-----------------------------------------------------------------------
\newcommand{\Ab}{\mathbf{A}}
\newcommand{\Bb}{\mathbf{B}}
\newcommand{\Cb}{\mathbf{C}}
\newcommand{\Db}{\mathbf{D}}
\newcommand{\Eb}{\mathbf{E}}
\newcommand{\Fb}{\mathbf{F}}
\newcommand{\Gb}{\mathbf{G}}
\newcommand{\Xb}{\mathbf{X}}
\newcommand{\Yb}{\mathbf{Y}}
\newcommand{\Zb}{\mathbf{Z}}


%-----------------------------------------------------------------------
% Calligraphic Chancery letters
%-----------------------------------------------------------------------
\newcommand{\Ap}{\mathpzc{A}}
\newcommand{\Bp}{\mathpzc{B}}
\newcommand{\Cp}{\mathpzc{C}}
\newcommand{\Dp}{\mathpzc{D}}
\newcommand{\Ep}{\mathpzc{E}}
\newcommand{\Fp}{\mathpzc{F}}
\newcommand{\Gp}{\mathpzc{G}}
\newcommand{\Xp}{\mathpzc{X}}
\newcommand{\Yp}{\mathpzc{Y}}
\newcommand{\Zp}{\mathpzc{Z}}


%-----------------------------------------------------------------------
% Greek letters and variations
%-----------------------------------------------------------------------
\newcommand{\al}{\alpha}
\newcommand{\be}{\beta}
\newcommand{\ga}{\gamma}
\newcommand{\Ga}{\Gamma}
\newcommand{\de}{\delta}
\newcommand{\Del}{\Delta}
\newcommand{\la}{\lambda}
\newcommand{\La}{\Lambda}
\newcommand{\om}{\omega}
\newcommand{\Om}{\Omega}
\newcommand{\io}{i\omega}
\newcommand{\si}{\sigma}
\newcommand{\Si}{\Sigma}
\newcommand{\eps}{\varepsilon}
\newcommand{\ka}{\kappa}
\newcommand{\na}{\nabla}
\newcommand{\Th}{\Theta}


%-----------------------------------------------------------------------
% Math "accents"
%-----------------------------------------------------------------------
\newcommand{\ov}[1]{\overline{#1}}
\newcommand{\un}[1]{\underline{#1}}
\renewcommand{\t}[1]{\tilde{#1}}
\newcommand{\ti}[1]{\widetilde{#1}}
\newcommand{\tc}[1]{\tilde{\mathcal{#1}}}
\renewcommand{\a}[1]{\acute{#1}}
\newcommand{\h}[1]{\hat{#1}}
\newcommand{\wh}[1]{\widehat{#1}}
\newcommand{\hc}[1]{\hat{\mathcal{#1}}}
\newcommand{\ch}[1]{\check{#1}}


%-----------------------------------------------------------------------
% Special sets
%-----------------------------------------------------------------------
\newcommand{\Nz}{\N_0}
\newcommand{\Cl}{ {\mathbb{C}^-} }
\newcommand{\Ci}{ {\mathbb{C}^0} }
\newcommand{\Cr}{ {\mathbb{C}^+} }
\newcommand{\Ri}{\R\cup\{\infty\}}
\newcommand{\Cri}{\Ci\cup\Cr\cup\{\infty\}}
\newcommand{\Cii}{\Ci\cup\{\infty\}}
\newcommand{\Rpi}{[0,\infty]}
\newcommand{\Rii}{(-\i,\i)}
\newcommand{\Rp}{[0,\infty)}
\newcommand{\RpT}{[T,\infty)}
\newcommand{\Rn}{\R^{n}}
\newcommand{\Rnn}{\R^{n\times n}}
\newcommand{\Rm}[2]{\R^{#1\times #2}}


%-----------------------------------------------------------------------
% Lebesgue spaces and Hilbert spaces
%-----------------------------------------------------------------------
\newcommand{\hilb}{\mathscr{H}}
\newcommand{\Le}{\mathscr{L}}
\newcommand{\Lt}{\Le_{2}}
\newcommand{\Lo}{\Le_{1}}
\newcommand{\Lte}{\Le_{2e}}
\newcommand{\Li}{\Le_{\i}}
\newcommand{\Ht}{\mathcal{H}_{2}}
\newcommand{\Hinf}{\mathcal{H}_{\infty}}
\newcommand{\Kt}{\mathcal{K}_{2}}
\newcommand{\Lp}{\Le_p}
\renewcommand{\le}{{\ell}}
\newcommand{\lt}{\le_2}
\newcommand{\lte}{\le_{2e}}
\newcommand{\llt}{\ti \Le_{2}}
\newcommand{\llte}{\ti \Le_{2e}}
\newcommand{\lh}{\le(\hilb)}
\newcommand{\lth}{\le_2(\hilb)}
\newcommand{\lhn}[1]{\le(\hilb^{#1})}
\newcommand{\lthn}[1]{\le_2(\hilb^{#1})}
\newcommand{\blt}{\bm{\ell}_2}
\newcommand{\blte}{\bm{\ell}_{2e}}
\newcommand{\rhi}{RH_\infty}
\newcommand{\rli}{RL_\infty}


% RedeclareMathOperator command ------------------------------------------------
\makeatletter
\newcommand\RedeclareMathOperator{%
	\@ifstar{\def\rmo@s{m}\rmo@redeclare}{\def\rmo@s{o}\rmo@redeclare}%
}
% this is taken from \renew@command
\newcommand\rmo@redeclare[2]{%
	\begingroup \escapechar\m@ne\xdef\@gtempa{{\string#1}}\endgroup
	\expandafter\@ifundefined\@gtempa
	{\@latex@error{\noexpand#1undefined}\@ehc}%
	\relax
	\expandafter\rmo@declmathop\rmo@s{#1}{#2}}
% This is just \@declmathop without \@ifdefinable
\newcommand\rmo@declmathop[3]{%
	\DeclareRobustCommand{#2}{\qopname\newmcodes@#1{#3}}%
}
\@onlypreamble\RedeclareMathOperator
\makeatother%-------------------------------------------------------------------------


%-----------------------------------------------------------------------
% Math operators
%-----------------------------------------------------------------------
\RedeclareMathOperator{\Im}{Im} % Imaginary part
\RedeclareMathOperator{\Re}{Re} % Real part
\DeclareMathOperator{\grad}{grad} % Gradient
\RedeclareMathOperator{\div}{div}  % Divergence
\DeclareMathOperator{\rot}{rot}      % Rotarion
\DeclareMathOperator*{\sgn}{sgn}  % signum function
\DeclareMathOperator*{\sign}{sign} % signum function
\DeclareMathOperator{\sat}{sat}     % saturation
\DeclareMathOperator{\id}{id}        % identity
\DeclareMathOperator{\supp}{supp} % support of a function
\RedeclareMathOperator{\ker}{ker}   % kernel
\DeclareMathOperator{\im}{im}         % image
\DeclareMathOperator*{\esssup}{ess\,sup} % essential supremum
\DeclareMathOperator{\eim}{ess.im}         % essential image
\DeclareMathOperator{\End}{End}  % set of endomorphisms
\newcommand{\ek}{\End_\K}       
\DeclareMathOperator{\Hom}{Hom} % set of homomorphisms
\DeclareMathOperator{\GL}{GL}      % general linear group
\DeclareMathOperator{\GF}{GF}      % special linear group
\DeclareMathOperator{\card}{card}  % cardinality
\DeclareMathOperator{\diam}{diam} % diam
\DeclareMathOperator{\dist}{dist}      % distance
\DeclareMathOperator{\dime}{dim}     % dimension
\DeclareMathOperator{\codim}{codim} % codimension
\DeclareMathOperator{\co}{co}            % convex hull
\DeclareMathOperator{\rint}{rint}         % relative interior
\DeclareMathOperator{\aff}{aff}            % affine hull
\DeclareMathOperator{\spann}{span}  % linear span
\DeclareMathOperator{\qrint}{qrint}      % quasi relative interior
\DeclareMathOperator{\cone}{cone}     % cone generated by a set
\DeclareMathOperator{\psrint}{print}     % pseudo relative interior
\DeclareMathOperator{\core}{core}       % algebraic interior
\DeclareMathOperator{\lin}{lin}              % algebraic boundary?
\DeclareMathOperator{\lina}{lina}         % algebraic closure
\DeclareMathOperator{\inter}{int}         % interior
\DeclareMathOperator{\adj}{adj}           % algebraic adjoint of a matrix
\DeclareMathOperator*{\diag}{diag}      % diagonal matrix
\DeclareMathOperator*{\col}{col}    % col(A_1, .. , A_n) = (A_1^*, ... , A_n^*)^*
\DeclareMathOperator{\rk}{rk}             % rank of a matrix
\DeclareMathOperator{\He}{He}          % He(A) = A + A^*
\DeclareMathOperator{\eig}{eig}           % eigenvalues
\DeclareMathOperator{\tr}{tr}               % trace
\DeclareMathOperator{\trace}{trace}     % trace
\newcommand{\lama}{\la_{\mathrm{max}}} % maximal eigenvalue
\newcommand{\lami}{\la_{\mathrm{min}}}   % minimal eigenvalue
\DeclareMathOperator{\kat}{Kat}
\DeclareMathOperator{\rad}{rad}
\DeclareMathOperator{\slope}{slope}
\DeclareMathOperator{\z}{z}
\DeclareMathOperator{\dz}{dz}


%-----------------------------------------------------------------------
% Math symbols
%-----------------------------------------------------------------------
\newcommand{\cge}{\succcurlyeq}      
\newcommand{\cle}{\preccurlyeq}
\newcommand{\cl}{\prec}
\newcommand{\cg}{\succ}
\newcommand{\lto}{\longrightarrow}  %lon arrow for \to
\newcommand{\too}{\searrow}
\newcommand{\tou}{\nearrow}
\newcommand{\kron}{\otimes}         % kronecker product
\renewcommand{\i}{\infty}
\newcommand{\bul}{\bullet}
\newcommand{\epro}{ \mbox{\rule{2mm}{2mm}} }
\newcommand{\purp}{\bot}            % for basismatrices of a kernel
\newcommand{\obot}{\mbox{\ \footnotesize$\stackrel{\bot}{\oplus}$\ }} % orthogonal sum
\newcommand{\Delf}{\mathbf{\Delta}}


%-----------------------------------------------------------------------
% Other
%-----------------------------------------------------------------------
\newcommand{\opt}{\mathrm{opt}} % subscript for optimal values
\newcommand{\dis}{\displaystyle}
\newcommand{\st}{\ |\ }                     % such that
\newcommand{\sst}[2]{\left\{#1 \, \middle| \; #2 \right\}} % such that
\newcommand{\alert}[1]{{\bf #1}}

% scalar products
\newcommand{\ab}[1]{\langle #1 \rangle}
\newcommand{\ip}[1]{\langle #1 \rangle}
\newcommand{\skp}[1]{\langle #1 \rangle}
\newcommand{\lskp}[1]{\left\langle #1 \right\rangle}

% minimize
\newcommand{\minimize}[2]{
\begin{array}{ll}
\mbox{minimize}    &\ \  #1 \\[1ex]
\mbox{subject\ to} &\ \  #2
\end{array} }

% maximize
\newcommand{\maximize}[2]{
\begin{array}{ll}
\mbox{maximize}    &\ \  #1 \\
\mbox{subject\ to} &\ \  #2
\end{array} }


%-----------------------------------------------------------------------
% Matrices etc
%-----------------------------------------------------------------------
\newcommand{\arr}[2]{\begin{array}{#1}#2\end{array}}
\newcommand{\tab}[2]{\begin{tabular}{#1}#2\end{tabular}}
\newcommand{\mas}[2]{\left[\begin{array}{#1}#2\end{array}\right]}
\newcommand{\mat}[2]{\left(\begin{array}{#1}#2\end{array}\right)}
\newcommand{\matl}[4]{\left#1\begin{array}{#2}#3\end{array}\right#4}
\newcommand{\dmas}[2]{\mas{ccc}{#1& & 0\\[-1.5ex] & \ddots &\\[-1.5ex] 0 & & #2}}
\newcommand{\dmat}[2]{\mat{ccc}{#1& & 0\\[-1.5ex] & \ddots &\\[-1.5ex] 0 & & #2}}
\newcommand{\smat}[1]{\left(\begin{smallmatrix}#1\end{smallmatrix}\right)}
\newcommand{\smas}[1]{\left[\begin{smallmatrix}#1\end{smallmatrix}\right]}
\newcommand{\mass}[2]{\scriptsize
	\renewcommand{\arraystretch}{0.5}
	\arraycolsep=3pt \mas{#1}{#2}}
\newcommand{\as}[1]{\renewcommand{\arraystretch}{#1}} %adjust size
\newcommand{\hl}{\\\hline}  % horizontal line and break
\newcommand{\hdl}{\\\hdashline} % horizontal dashed line and break


%-----------------------------------------------------------------------
% Math environments
%-----------------------------------------------------------------------
\newcommand{\aln}[1]{\noindent\begin{align*}#1\end{align*}}
\newcommand{\mun}[1]{\vspace*{-1ex}\noindent
	\begin{multline*}#1\end{multline*}}
\newcommand{\mul}[1]{\vspace*{-1ex}\noindent
	\begin{multline}#1\end{multline}}
\newcommand{\eqn}[1]{$$#1$$}
\newcommand{\equ}[1]{\begin{equation}#1\end{equation}}
\newcommand{\eql}[2]{\begin{equation}\label{#1}#2\end{equation}}
\newcommand{\eqns}[1]{\vspace*{-2ex}$$#1$$\vspace*{0ex}}
\newcommand{\gan}[1]{\begin{gather*}#1\end{gather*}}
\newcommand{\ear}[1]{\begin{eqnarray}#1\end{eqnarray}}
\newcommand{\ean}[1]{\begin{eqnarray*}#1\end{eqnarray*}}
\renewcommand{\r}[1]{\eqref{#1}} % shortcut for referencing
\renewcommand{\l}[1]{\label{#1}}  % shortcut for setting labels
\newcommand{\te}[1]{\text{\ \ #1\ \ }} % for text in math enironments
\newcommand{\teq}[1]{\quad\text{#1}\quad} % for text in math enironments

%-----------------------------------------------------------------------
% Theorem like environments
%-----------------------------------------------------------------------
\newtheorem{theo}{Theorem}%[section]
\newtheorem{prop}[theo]{Proposition}
\newtheorem{hypo}[theo]{Hypothesis}
\newtheorem{lemm}[theo]{Lemma}
\newtheorem{coro}[theo]{Corollary}
\newtheorem{defi}[theo]{Definition}
\newtheorem{assu}[theo]{Assumption}
\newtheorem{rema}[theo]{Remark}
\newtheorem{exam}[theo]{Example}
\newtheorem{prob}[theo]{Problem}

\newenvironment{proof}[1][Proof]{%\vspace{1.5ex}
	\bf #1:\\ \rm}
{\hfill \footnotesize{$\blacksquare$}\vspace{2ex}}

\newcommand{\hypothesis}[1]{\begin{hypo}#1\end{hypo}}
\newcommand{\theorem}[1]{\begin{theo}#1\end{theo}}
\newcommand{\corollary}[1]{\begin{coro}#1\end{coro}}
\newcommand{\lemma}[1]{\begin{lemm}#1\end{lemm}}
\newcommand{\definition}[1]{\begin{defi}#1\end{defi}}
\newcommand{\assumption}[1]{\begin{assu}#1\end{assu}}
\newcommand{\example}[1]{\begin{exam}#1\end{exam}}
\newcommand{\remark}[1]{\begin{rema}#1\end{rema}}
\newcommand{\proo}[1]{\begin{proof}#1\end{proof}}


%-----------------------------------------------------------------------
% Abbreviations, lists and tabulars
%-----------------------------------------------------------------------
\newcommand{\enu}[1]{\vspace*{-1ex}\begin{enumerate}#1\end{enumerate}}
\newcommand{\ite}[1]{\vspace*{-1ex}\begin{list}{$\bullet$}{\leftmargin2.5ex}\itemsep0ex #1\end{list} }
\newcommand{\ited}[2]{\vspace*{-1ex}\begin{list}{$\bullet$}{\leftmargin#1}\itemsep0ex #2\end{list} }
\newcommand{\itemi}[1]{\begin{itemize}#1\end{itemize} }
\newcommand{\des}[1]{\begin{description}#1\end{description} }
\newcommand{\ann}[1]{ \scriptsize \noindent \textbf{Annotation. }{\leftskip=.3in #1 \begin{flushright}$\star$\end{flushright}} \normalsize}
\newcommand{\tabu}[2]{\begin{tabular}{#1}#2\end{tabular}}
\newcommand{\tabi}[1]{\begin{tabbing}#1\end{tabbing}}
% enumerate numbering
\renewcommand{\labelenumi}{\arabic{enumi}.}
\renewcommand{\labelenumii}{\alph{enumii})}
\newcommand*\fixitem {\item[]%
	\hskip-\leftmargin\labelitemi\hskip\labelsep}


%\setstretch{1.5} %Zeilenabstand (1,5)

\parindent 0mm
\newcommand{\bflabel}[1]{\normalfont{\normalsize{#1}}\hfill}

\pagestyle{plain}
\pagestyle{fancy}
%\cfoot{\thepage}

 \setcounter{secnumdepth}{3}
 \setcounter{tocdepth}{3}


\begin{document}


\clearpage
\setlength{\hoffset}{-15mm}

\pagenumbering{Roman}

%%%%%%%%%%%%%%%%%%%%%%%%%%%%% Titelblatt %%%%%%%%%%%%%%%%%%%%%%%%%%%%%%%%%%
\begin{titlepage}
\newfont{\smc}{cmcsc10 at 16pt}

\begin{center}
\smc
Lehrstuhl fuer Mathematische Systemtheorie\\
Universitaet Stuttgart\\

\vspace{2.5cm}

\Large \textbf{BACHELORARBEIT}

\vspace{2cm}

\large\textbf{im Bachelorstudiengang \\Mathematik}\\

\vspace{3cm}

\fbox{\parbox{\columnwidth}{
\begin{center}
\large \textbf{TITEL DER ARBEIT}
\end{center}}}


\vspace{3cm}
\begin{table}[h]
	\centering
		\begin{tabular}{ll}
Erstgutachter/in:  & Prof. Dr. Carsten W. Scherer\\
Zweitgutachter/in: & ....\\
		\end{tabular}
\end{table}

\vspace{3cm}
\normalsize vorgelegt von
\end{center}

\vfill
\parbox{5cm}{\footnotesize
            Name\\
            Strasse\\
            PLZ + Ort\\
            E-mail}
\hfill
\parbox{5cm}{\footnotesize
            Studienfach\\
            Fachsemester\\
            Matrikelnummer\\
            Abgabetermin: XX.XX.201X}

\end{titlepage}

\clearpage
\setlength{\hoffset}{0mm}
\newpage
\thispagestyle{empty}
\mbox{}
\newpage

%%%%%%%%%%%%%%%%%%%%%%%%%%%%% Inhaltsverzeichnis %%%%%%%%%%%%%%%%%%%%%%%%%%%%%%%%%%

\thispagestyle{empty}

\tableofcontents

%%%%%%%%%%%%%%%%%%%%%%%%%%%%% Abbildungsverzeichnis %%%%%%%%%%%%%%%%%%%%%%%%%%%%%%%
\newpage
%\pagenumbering{Roman}
\setcounter{page}{1}
\addcontentsline{toc}{section}{Abbildungsverzeichnis\vspace{0pt}}
\listoffigures

%%%%%%%%%%%%%%%%%%%%%%%%%%%%% Tabellenverzeichnis %%%%%%%%%%%%%%%%%%%%%%%%%%%%%%%%%
\newpage
\addcontentsline{toc}{section}{Tabellenverzeichnis\vspace{0pt}}
\listoftables

%%%%%%%%%%%%%%%%%%%%%%%%%%%%% Abk�rzungsverzeichnis %%%%%%%%%%%%%%%%%%%%%%%%%%%%%%%
\newpage
\section*{Abkuerzungsverzeichnis} \markboth{ABKÜRZUNGSVERZEICHNIS}{ABKÜRZUNGSVERZEICHNIS}
%Bitte alphabetisch ordnen!
\addcontentsline{toc}{section}{Abkürzungsverzeichnis\vspace{0pt}}
\rule[0pt]{0mm}{10pt} %\rule[Offset]{Breite}{H�he}
\begin{acronym}[Musterdingsbums]
\setlength{\itemsep}{-\parsep}
	% A
	\acro{O.B.d.A}{...}
	% B
	% C
	% D
	% E
	% F
	% G
	% H
	% I
	% J
	% K
	% L
	% M
	% N
	% O
	% P
	% Q
	% R
	% S
	% T
	% U
	% V
	% W
	% X
	% Y
	% Z
\end{acronym}

%%%%%%%%%%%%%%%%%%%%%%%%%%%%% Symbolverzeichnis %%%%%%%%%%%%%%%%%%%%%%%%%%%%%%%
\newpage
\section*{Symbolverzeichnis} \markboth{SYMBOLVERZEICHNIS}{SYMBOLVERZEICHNIS}
%Die hier verwendeten Symbole stellen eine beispielhafte Auswahl dar.
\addcontentsline{toc}{section}{Symbolverzeichnis\vspace{0pt}}
\rule[0pt]{0mm}{10pt} %\rule[Offset]{Breite}{Höhe}
\begin{acronym}[Musterdingsbums]
\setlength{\itemsep}{-\parsep}
	\acro{G}[$G(s)$]{Übertragungsfunktion eines LTI Systems}
	\acro{R}[$(A,B,C,D)$]{Realization des Systems im Zustandsraum}
	%\acro{C}[$C(\cdot)$]{Kostenfunktion}
\end{acronym}

%%%%%%%%%%%%%%%%%%%%%%%%%%% Bsp. f�r das Einf�gen von Graphiken (extern) %%%%%%%%%%%%%%%%%%%%%%%
% \begin{figure}[h!]
% \includegraphics[width=\textwidth]{Pfad der Datei/Dateiname.Endung}\vspace{-0.5cm}
% \caption{Bildunterschrift}
% \end{figure}

%%%%%%%%%%%%%%%%%%%%%%%%%% Beginn des Textteils %%%%%%%%%%%%%%%%%%%%%%%%%%%%%%%

\fancyhf{}				% Alle Felder loeschen
\renewcommand{\sectionmark}[1]		% Schriftform f�r \section = Kapt�lchen
	{\markboth{\sc \thesection{} #1}{}}
%\fancyhead[LO]{\leftmark}		% immer links oben Kapitel
%\fancyhead[RE]{\rightmark}		% und rechts oben Unterkapitel
%\fancyfoot[RE,RO]{\thepage}
\addtolength{\headheight}{2.0pt}


\newpage
\pagenumbering{arabic}
%Einleitung
\section{Einleitung}
Die Einleitung gliedert sich in den 3 Unterpunkten:

  \subsection{Motivation} Die Einleitung muss den Leser zum weiterlesen bewegen. Hilfreich sind hier manchmal konkrete Beispiele für Anwendungen.
   Die Problemstellung motivieren, erklaren warum die Problemstellung behandelt wird und wofür eine Lösung des Problems wichtig ist.
  \subsection{Zielsetzung} Motivieren Sie Ihr Thema und geben Sie deutlich die Problemstellung an!
  Formulieren Sie zu Beginn der Arbeit die Forschungsfrage/n, die Sie in der Arbeit beantworten wollen. In der Regel haben Sie eine Hauptfrage. Zur Unterstützung können dann verschiedene Teilfragen formuliert werden.
  \subsection{Aufbau der Arbeit} Am Ende der Einleitung muss die Struktur der Arbeit kurz erläutert werden.
  Die Arbeit ist in $x$ Kapiteln aufgebaut. Im ersten Kapitel wird die Problemstellung ausführlich formuliert. Außerdem werden die in der Literatur vorhandene Verfahren zur Lösung des Problems diskutiert.

   Das zweite und dritte Kapitel bilden zusammen der Kern dieser Arbeit. Dort wird das Lösungsansatz vorgestellt und erklärt.....
   Im letzten Kapitel werden die vorgestellten Algorithmen und Ansätze durch Simulationsergnisse validiert....



%\begin{figure}[h]
%	\centering
%		\includegraphics[width=0.65\textwidth]{Grafik11}
%	\caption{Beispiel 1}
%	\label{fig:Grafik11}
%\end{figure}

\newpage
\section{Mathematical model}
In order to derive the equations of motion for the model, the Lagrangian approach is used. The model is largely based on the model used in (Seb13) and uses refinments 
\subsection{Assumptions}



\newpage
\section{Stand der Technik}
Es ist wichtig, dass die Literatur studiert wird. In der Literatur angebotene Lösungsansätze sollten kurz diskutiert und besprochen werden und die Wahl des Verfahrens für die Bachelorarbeit sollte begründet werden.
%Erkl�ren, wie bisher an die Problemstellung herangegangen wurde und wichtige vorhandene L�sungsans�tze kurz erkl�ren und diskutieren (Vor und Nachteile).\\
\newpage


\section{Layout der Arbeit}

\subsection{Die Gliederung}

Eine gute Strukturierung ist wichtig! Die Arbeit muss einen roten Faden haben und Sie sollten Schwerpunkte setzen. Es geht nicht darum so viel wie möglich zu schreiben, sonder so viel wie nötig!

\subsubsection{Regel 1}
Die Gliederung der Arbeit sollte nicht zu tief sein. In der Regel sind mehr als 3 Gliederungsebenen nicht notwendig.

\subsubsection{Regel 2}
Eine Gliederungsebene sollte immer mindestens zwei Unterkapitel enthalten. Jede Arbeit enthält eine Einleitung und ein zusammenfassendes Schlusskapitel.

\subsection{Quellenangaben}
Quellenangaben sollten im Text erscheinen unter Angabe des/der Autors/Autoren und der Jahreszahl der Publikation. Wichtig ist, dass die Zuordnung der Quelle zum Literaturverzeichnis \textbf{eindeutig} ist! Zitieren Sie möglichst immer die Originalquelle! Sinngemäße Zitate werden mit "'Vgl."' eingeleitet. Das Literaturverzeichnis \textbf{muss} vollständig sein! Alle im Text aufgeführten Quellen müssen sich im Literaturverzeichnis wieder finden lassen und alle im Verzeichnis aufgeführten Quellen müssen mindestens einmal im Text erwähnt werden.

%Zusammenfassung /Res�mee /Ausblick
\section{Zusammenfassung und Ausblick}%\sectionmark{Zusammenfassung}

Hier können auch Schlußfolgerungen präsentiert und ein Ausblick gegeben werden.

%%%%%%%%%%%%%%%%%%%%%%%%%%% Literaturverzeichnis %%%%%%%%%%%%%%%%%%%%%%%%%%%%%%%%
\newpage
\newpage\pagenumbering{Roman}
\setcounter{page}{5}

\begin{thebibliography}{99} \sectionmark{LITERATURVERZEICHNIS}
\addcontentsline{toc}{section}{Literaturverzeichnis}

\harvarditem{Lessard}{2009}{L. Lessard, S. Lall}
 L. Lessard, S. Lall.
  Proceedings of the IEEE Conference on Decision and Control, p. 1621--1626, 2009.


%\harvarditem{Mitroff}{1974}{Mitroff} II Mitroff, F Betz, LR Pondy, F Sagasti.
%On managing science in the systems age: two schemes for the study of science as a whole systems
%phenomenon. Interfaces 4, (1974), 46-- ??.

\end{thebibliography}



%%%%%%%%%%%%%%%%%%%%%%%%%%% APPENDIX %%%%%%%%%%%%%%%%%%%%%%%%%%%%%%%%%%%%%%%%%%%%
\newpage\clearpage
\addcontentsline{toc}{section}{Anhang}
\begin{appendix}
\section{Beweise}\sectionmark{Beweise}
\subsection{Beweis 1}
\begin{eqnarray}
1+1 = 2
\end{eqnarray}
\end{appendix}



%%%%%%%%%%%%%%%%%%%%%%%%%%% Eidesstattliche Erkl�rung %%%%%%%%%%%%%%%%%%%%%%%%%%%%%%%%
\newpage
\section*{Eidesstattliche Erklärung}\sectionmark{Eidesstattliche Erklärung}
\addcontentsline{toc}{section}{Eidesstattliche Erklärung}%\addtocontents{toc}{\vfill}
Ich erkläre, dass ich meine Bachelor-Arbeit [\emph{Titel der Arbeit}] selbstständig und ohne Benutzung anderer als der angegebenen Hilfsmittel angefertigt habe und dass ich alle Stellen, die ich wörtlich oder sinngemäß aus Veröffentlichungen entnommen habe, als solche kenntlich gemacht habe. Die Arbeit hat bisher in gleicher oder ähnlicher Form oder auszugsweise noch keiner Prüfungsbehörde vorgelegen. \\
%\\
%Ich versichere, dass die eingereichte schriftliche Fassung der auf dem beigef�gten Medium gespeicherten Fassung entspricht.[\emph{Dieser Zusatz ist nur erforderlich, wenn die Abschlussarbeit auch als Datei abgegeben worden ist.}]\\\\\\
\noindent Stuttgart, den XX.XX.2013
\begin{flushright}
$\overline{~~~~~~~~~\mbox{(Name des Kandidaten)}~~~~~~~~~}$
\end{flushright}
\end{document} 